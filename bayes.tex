\documentclass{article}
% Allow the usage of utf8 characters
\usepackage[utf8]{inputenc}
% Uncomment the following line to allow the usage of graphics (.png, .jpg)
\usepackage{graphicx}

% Start the document
\begin{document}

% Create a new 1st level heading
\section{Bayes Classifiers}

\subsection{a)}

Probabilities needed for joint Bayes classifier can be found by counting the number of occurances of each possible x1 x2 combination ([0,0] [0,1] [1,0] [1,1]). Then finding how many of each of these assosiate with each class. E.g.

Where $$ y=0 $$

$$ P(y|x) = \frac{1}{4}\ ,\ x = [0,0] $$

$$ P(y|x) = \frac{1}{4}\ ,\ x = [0,1] $$

$$ P(y|x) = \frac{3}{3}\ ,\ x = [1,0] $$

$$ P(y|x) = \frac{3}{5}\ ,\ x = [1,1] $$

Where $$ y=1 $$

$$ P(y|x) = \frac{3}{4}\ ,\ x = [0,0] $$

$$ P(y|x) = \frac{3}{4}\ ,\ x = [0,1] $$

$$ P(y|x) = \frac{0}{3}\ ,\ x = [1,0] $$

$$ P(y|x) = \frac{2}{5}\ ,\ x = [1,1] $$

Therefore, the complete class predictions on the test set looks as follows:

\begin{table}[h]
\begin{tabular}{lllll}
 x1  &  x2  &  P(y=0|x)  &  P(y=1|x)  &  y-hat \\
   0  &   1  &     25\%    &     75\%    &    1 \\
   1  &   0  &    100\%    &      0\%    &    0 \\
   1  &   1  &     60\%    &     40\%    &    0  
\end{tabular}
\end{table}

\subsection{(b)}

To create a Naive Bayes classifier first the probabilities of each class will be needed:

$$ P(y=0) = \frac{8}{16} $$

$$ P(y=1) = \frac{8}{16} $$

Next, the probability that each x could be in each class:

\begin{table}[h]
\begin{tabular}{lllll}
       &   y=0 &  y=1 \\
   x1  &  6/8  &  2/8\\
   x2  &  4/8  &  5/8
\end{tabular}
\end{table}

To classify the test set P(x|y) is calculated for each $x_1,\ x_2$ combination seen in the test set, on each class

$$ P(x=[0,1]\ |\ y=0) = P(x_1\ |\ y)\cdot P(x_2\ |\ y)$$

$$ = \left(1 - \frac{6}{8}\right)\cdot \left(\frac{4}{8}\right) $$

$$ = \frac{1}{8} $$

$$ P(x=[0,1]\ |\ y=1) = P(x_1\ |\ y)\cdot P(x_2\ |\ y)$$

$$ = \left(1 - \frac{2}{8}\right)\cdot \left(\frac{5}{8}\right) $$

$$ = \frac{15}{32} $$

$$ P(x=[1,0]\ |\ y=0) = P(x_1\ |\ y)\cdot P(x_2\ |\ y)$$

$$ = \left(\frac{6}{8}\right)\cdot \left(1 - \frac{4}{8}\right) $$

$$ = \frac{3}{8} $$

$$ P(x=[1,0]\ |\ y=1) = P(x_1\ |\ y)\cdot P(x_2\ |\ y)$$

$$ = \left(\frac{2}{8}\right)\cdot \left(1 - \frac{5}{8}\right) $$

$$ = \frac{3}{32} $$

$$ P(x=[1,1]\ |\ y=0) = P(x_1\ |\ y)\cdot P(x_2\ |\ y)$$

$$ = \left(\frac{6}{8}\right)\cdot \left(\frac{4}{8}\right) $$

$$ = \frac{3}{8} $$

$$ P(x=[1,1]\ |\ y=1) = P(x_1\ |\ y)\cdot P(x_2\ |\ y)$$

$$ = \left(\frac{2}{8}\right)\cdot \left(\frac{5}{8}\right) $$

$$ = \frac{5}{32} $$

Next the P(x) is calculated with the formula

$$ P(x) = \sum_i P(x|y_i)\cdot P(y_i) $$

on all test data values of x.

$$ P(x = [0,1]) = \left(\frac{1}{8} \cdot \frac{1}{2}\right) + \left(\frac{15}{32} \cdot \frac{1}{2}\right) $$

$$ = \frac{19}{64} $$

$$ P(x = [1,0]) = \left(\frac{3}{8} \cdot \frac{1}{2}\right) + \left(\frac{3}{32} \cdot \frac{1}{2}\right) $$

$$ = \frac{15}{64} $$

$$ P(x = [1,1]) = \left(\frac{3}{8} \cdot \frac{1}{2}\right) + \left(\frac{5}{32} \cdot \frac{1}{2}\right) $$

$$ = \frac{17}{64} $$

Finally, the probability of y given each x value can be found by using Bayes rule:

$$ P(y|x) = \frac{P(x|y) \cdot P(y)}{P(x)} $$

On x=[0,1]:

$$ P(y=0|x=[0,1]) = \frac{P(x|y) \cdot P(y)}{P(x)} $$

$$ = \frac{(1/8) \cdot (1/2)}{19/64} $$

$$ \approx 21\% $$

Therefore:

$$ P(y=1|x=[0,1]) \approx 79\% $$

And on x=[1,0]:

$$ P(y=0|x=[1,0]) = \frac{P(x|y) \cdot P(y)}{P(x)} $$

$$ = \frac{(3/8) \cdot (1/2)}{15/64} $$

$$ \approx 80\% $$

Therefore:

$$ P(y=1|x=[0,1]) \approx 20\% $$

And on x=[1,1]:

$$ P(y=0|x=[1,1]) = \frac{P(x|y) \cdot P(y)}{P(x)} $$

$$ = \frac{(3/8) \cdot (1/2)}{17/64} $$

$$ \approx 71\% $$

Therefore:

$$ P(y=1|x=[0,1]) \approx 29\% $$

Therefore, the complete class predictions on the test set using Naive Bayes looks as follows:
\begin{table}[h]
\begin{tabular}{lllll}
  x1  &  x2  &  P(y=0|x)  &  P(y=1|x)  &  y-hat \\
   0  &   1  &     21\%    &     79\%    &    1 \\
   1  &   0  &     80\%    &     20\%    &    0 \\
   1  &   1  &     71\%    &     29\%    &    0
\end{tabular}
\end{table}




\end{document}